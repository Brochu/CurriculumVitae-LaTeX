%% start of file `template.tex'.
%% Copyright 2006-2013 Xavier Danaux (xdanaux@gmail.com).
%
% This work may be distributed and/or modified under the
% conditions of the LaTeX Project Public License version 1.3c,
% available at http://www.latex-project.org/lppl/.


\documentclass[11pt,arial]{moderncv}        % possible options include font size ('10pt', '11pt' and '12pt'), paper size ('a4paper', 'letterpaper', 'a5paper', 'legalpaper', 'executivepaper' and 'landscape') and font family ('sans' and 'roman')

% moderncv themes
\moderncvstyle{casual}                             % style options are 'casual' (default), 'classic', 'oldstyle' and 'banking'
\moderncvcolor{blue}                               % color options 'blue' (default), 'orange', 'green', 'red', 'purple', 'grey' and 'black'
%\renewcommand{\familydefault}{\sfdefault}         % to set the default font; use '\sfdefault' for the default sans serif font, '\rmdefault' for the default roman one, or any tex font name
%\nopagenumbers{}                                  % uncomment to suppress automatic page numbering for CVs longer than one page

% character encoding
\usepackage[utf8]{inputenc}                       % if you are not using xelatex ou lualatex, replace by the encoding you are using
%\usepackage{CJKutf8}                              % if you need to use CJK to typeset your resume in Chinese, Japanese or Korean

% adjust the page margins
\usepackage[scale=0.75]{geometry}
%\usepackage[latin1]{inputenc}
\usepackage[frenchb]{babel}
%\setlength{\hintscolumnwidth}{3cm}                % if you want to change the width of the column with the dates
%\setlength{\makecvtitlenamewidth}{10cm}           % for the 'classic' style, if you want to force the width allocated to your name and avoid line breaks. be careful though, the length is normally calculated to avoid any overlap with your personal info; use this at your own typographical risks...

% personal data
\name{Alexandre}{Brochu}
\address{1925 rue Émile-Martineau}{H7N 0B2 Laval}{QC}
% optional, remove / comment the line if not wanted; the optional "type" of the phone can be "mobile" (default), "fixed" or "fax"
\phone[mobile]{(819) 209-9772}
\email{alexbrochu1@gmail.com}
\homepage{brochu.github.io}
\social[linkedin][ca.linkedin.com/in/alexandrebrochu]{Alexandre Brochu}
\social[github]{Brochu}                             
%\quote{Objectif: Obtenir un emploi dans le domaine de la programmation informatique plus spécifiquement en programmation de jeux vidéo pour mettre en pratique mes études en informatique.}

% to show numerical labels in the bibliography (default is to show no labels); only useful if you make citations in your resume
%\makeatletter
%\renewcommand*{\bibliographyitemlabel}{\@biblabel{\arabic{enumiv}}}
%\makeatother
%\renewcommand*{\bibliographyitemlabel}{[\arabic{enumiv}]}% CONSIDER REPLACING THE ABOVE BY THIS

% bibliography with mutiple entries
%\usepackage{multibib}
%\newcites{book,misc}{{Books},{Others}}
%----------------------------------------------------------------------------------
%            content
%----------------------------------------------------------------------------------
\begin{document}
\makecvtitle
\pagestyle{empty}
\thispagestyle{plain}

\section{Éducation}
\cventry{2012-2015}{Bacc en informatique}{Université de Sherbrooke}{Sherbrooke (Québec)}{}
    {
            \begin{itemize}
                \item Moyenne générale finale de 4.06
                \item Programmation de rendu
                \item Intelligence artificielle
                \item Programmation orientée objet
            \end{itemize}
    }  % arguments 3 to 6 can be left empty
\cventry{2009-2012}{Technique en informatique}{Cégep de l'Outaouais}{Gatineau (Québec)}{}{}
% NOT SURE ABOUT THIS, MAYBE KEEP THIS??
% \cventry{2004-2009}{Diplôme d'études secondaire}{Polyvalente de l'érablière}{Gatineau (Québec)}{}{ Programme micro informatique (PMI)}

\section{Expériences de travail}
\cventry{Nov.~2019\\present}{Programmeur Unreal}{Behaviour Interactive}{Montréal}{QC}{
\begin{itemize}
	\item Travailler avec l'engin de jeu Unreal Engine 4
	\item Écrire la logique du jeu et des outils (utilisant C++ et 'blueprints')
	\item Publier un jeu sous une nouvelle plateforme récente (Stadia)
	\item Pratiquer de nouveau des aspects de la programmation graphique
	\item Faire part d'un projet à succès avec des mises à jour
\end{itemize}}
\cventry{Juin~2016\\Nov.~2019}{Programmeur Unity}{Behaviour Interactive}{Montréal}{QC}{
\begin{itemize}
	\item Travailler avec l'engin de jeux Unity3D
	\item Écrire la logique du jeu (AI, Animation, UI, Systems and Backend)
    \item Planifier la logique de jeu et des systèmes
	\item Publier des jeux sous multiples plateformes (Android, iOS)
	\item Pratiquer le domaine de la programmation graphique
\end{itemize}}
\cventry{Juillet~2015\\Juin~2016}{Programmeur Unity}{Fuel Industries}{Ottawa}{ON}{
\begin{itemize}
	\item Travail avec le moteur Unity3D
	\item Modifier le code du jeu directement (intelligence artificielle, animation, interface usager)
	\item Ajouter des fonctionnalités au jeu d'après des spécifications
	\item Régler des problèmes identifiés par une équipe de testeurs
	\item Développement d'un jeu sur une plateforme mobile
\end{itemize}}
\cventry{mai~2014\\août~2014}{Programmeur généraliste}{Square Enix Montreal}{Montréal}{QC}{
\begin{itemize}
	\item Travail avec le moteur Unity3D
	\item Modifier le code du jeu directement (intelligence artificielle, animation, interface usager)
	\item Ajouter des fonctionnalités au jeu d'après des spécifications
	\item Régler des problèmes identifiés par une équipe de testeurs
	\item Développement d'un jeu sur une plateforme mobile
\end{itemize}}
\cventry{janv.~2013\\avr.~2013 \\ sept.~2013\\déc.~2013}{Développeur Web}{Environnement Canada}{Gatineau}{QC}{
\begin{itemize}
	\item Travail en grande partie autonome
	\item Développer une nouvelle version d'un outils pour aider à la productivité des employés
	\item Planifier une implentation d'une base de données
	\item Écrire le code d'un outil de publication web (asp.net, MVC)
	\item Aider à la gestion de serveurs web
\end{itemize}}
%\cventry{janv.~2012\\août~2012}{Développeur logiciel}{STaCS DNA}{Ottawa}{ON}{
%\begin{itemize}
%	\item Travail avec une équipe de testing
%	\item Ajout de fonctionnalités dans un logiciel
%	\item Modifier le logiciel en se basant sur des spécifications
%	\item Régler des problèmes soulignés par les clients
%	\item Modifier et maintenir une grande base de données SQL
%	\item Stage complètement en anglais
%\end{itemize}}

\section{Compétences techniques}
\cvitem{Systèmes d'exploitation}{Microsoft, Linux, MacOS, Android}
\cvitem{Langages}{C++, C\#, Javascript, Ruby, Python, Racket, Java}
\cvitem{Outils de développement}{Visual Studio, Visual Studio Code, Rider for Unreal Engine, Vim, CMake, GDB}
\cvitem{Gestion de version}{Git (Command line, Sourcetree, Fork), Github, Bitbucket, GitLab, Perforce, SVN, TeamFoundation}
\cvitem{Moteurs de jeu}{Unity 3D, Unreal Engine 4, DromEd}
% À voir si j'ai besoin de plus
% \cvitem{category 5}{XXX, YYY, ZZZ}
% \cvitem{category 6}{XXX, YYY, ZZZ}

\section{Prix et distinctions}
\cvlistitem{Deux différentes mentions d'excellence pour les sessions été 2013 et hiver 2015 à l'Université de Sherbrooke suite à l'obtention d'une note moyenne finale de 4.3 (A+) pour ces sessions.}
\cvlistitem{Meilleur rendement physique 536}
\cvlistitem{Meilleur rendement chimie 584}
% \cvlistitem{Item 3. This item is particularly long and therefore normally spans over several lines. Did you notice the indentation when the line wraps?}

\section{Projets personnels et universitaires}
\cvlistdoubleitem{Moteur de rendu en tracé de rayons}{Implémentation d'un algorithme d'apprentissage par récompenses}
\cvlistdoubleitem{Visualisateur du système solaire (OpenGL)}{Génétateur de fractales parallèle}
\cvlistdoubleitem{Travail sur des projets personels pour explorer les possibilités des engins de jeux populaires}{Création d'un jeu basé sur des mécaniques observées dans un jeu existant}

\section{Intérêts}
\cvitem{Jeux vidéo}{J'aime bien jouer aux jeux vidéo dans mes temps libre d'une part pour l'expérience et aussi pour comprendre ce qui rend un jeu intéressant}
\cvitem{Ordinateur}{Au cours de mes études, j'ai bâti des ordinateurs pour des proches ainsi que moi-même}
\cvitem{Mathématiques}{Depuis mes cours de mathématiques à l'Université, J'aime comprendre les concepts mathématiques qui sont à la base de la programmation}
\cvitem{Musique}{J'ai appris à jouer de la guitare électrique de façon autodidacte. J'ai beaucoup développé ma curiosité et ma capacité à apprendre par moi-même avec cette expérience.}

% Publications from a BibTeX file without multibib
%  for numerical labels: \renewcommand{\bibliographyitemlabel}{\@biblabel{\arabic{enumiv}}}% CONSIDER MERGING WITH PREAMBLE PART
%  to redefine the heading string ("Publications"): \renewcommand{\refname}{Articles}
% \nocite{*}
% \bibliographystyle{plain}
% \bibliography{publications}                        % 'publications' is the name of a BibTeX file

% Publications from a BibTeX file using the multibib package
%\section{Publications}
%\nocitebook{book1,book2}
%\bibliographystylebook{plain}
%\bibliographybook{publications}                   % 'publications' is the name of a BibTeX file
%\nocitemisc{misc1,misc2,misc3}
%\bibliographystylemisc{plain}
%\bibliographymisc{publications}                   % 'publications' is the name of a BibTeX file

\clearpage
%-----       letter       ---------------------------------------------------------
% recipient data
\recipient{Snowed In Studios}{Snowed In Studios\\123 TopKekStreet\\Ottawa (ON)}
\date{19 Mars 2015}
\opening{Dear Sir or Madam,}
\closing{Yours faithfully,}
\enclosure[fichier attaché]{curriculum vit\ae{}}          % use an optional argument to use a string other than "Enclosure", or redefine \enclname
\makelettertitle

Lorem ipsum dolor sit amet, consectetur adipiscing elit. Duis ullamcorper neque sit amet lectus facilisis sed luctus nisl iaculis. Vivamus at neque arcu, sed tempor quam. Curabitur pharetra tincidunt tincidunt. Morbi volutpat feugiat mauris, quis tempor neque vehicula volutpat. Duis tristique justo vel massa fermentum accumsan. Mauris ante elit, feugiat vestibulum tempor eget, eleifend ac ipsum. Donec scelerisque lobortis ipsum eu vestibulum. Pellentesque vel massa at felis accumsan rhoncus.

Suspendisse commodo, massa eu congue tincidunt, elit mauris pellentesque orci, cursus tempor odio nisl euismod augue. Aliquam adipiscing nibh ut odio sodales et pulvinar tortor laoreet. Mauris a accumsan ligula. Class aptent taciti sociosqu ad litora torquent per conubia nostra, per inceptos himenaeos. Suspendisse vulputate sem vehicula ipsum varius nec tempus dui dapibus. Phasellus et est urna, ut auctor erat. Sed tincidunt odio id odio aliquam mattis. Donec sapien nulla, feugiat eget adipiscing sit amet, lacinia ut dolor. Phasellus tincidunt, leo a fringilla consectetur, felis diam aliquam urna, vitae aliquet lectus orci nec velit. Vivamus dapibus varius blandit.

Duis sit amet magna ante, at sodales diam. Aenean consectetur porta risus et sagittis. Ut interdum, enim varius pellentesque tincidunt, magna libero sodales tortor, ut fermentum nunc metus a ante. Vivamus odio leo, tincidunt eu luctus ut, sollicitudin sit amet metus. Nunc sed orci lectus. Ut sodales magna sed velit volutpat sit amet pulvinar diam venenatis.

Albert Einstein discovered that $e=mc^2$ in 1905.
$$ e=\lim_{n \to \infty} \left(1+\frac{1}{n}\right)^n $$

\makeletterclosing
\end{document}
