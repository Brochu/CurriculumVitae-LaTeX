%% start of file `template.tex'.
%% Copyright 2006-2013 Xavier Danaux (xdanaux@gmail.com).
%
% This work may be distributed and/or modified under the
% conditions of the LaTeX Project Public License version 1.3c,
% available at http://www.latex-project.org/lppl/.


\documentclass[11pt,arial]{moderncv}        % possible options include font size ('10pt', '11pt' and '12pt'), paper size ('a4paper', 'letterpaper', 'a5paper', 'legalpaper', 'executivepaper' and 'landscape') and font family ('sans' and 'roman')

% moderncv themes
\moderncvstyle{casual}                             % style options are 'casual' (default), 'classic', 'oldstyle' and 'banking'
\moderncvcolor{blue}                               % color options 'blue' (default), 'orange', 'green', 'red', 'purple', 'grey' and 'black'
%\renewcommand{\familydefault}{\sfdefault}         % to set the default font; use '\sfdefault' for the default sans serif font, '\rmdefault' for the default roman one, or any tex font name
%\nopagenumbers{}                                  % uncomment to suppress automatic page numbering for CVs longer than one page

% character encoding
%\usepackage[utf8]{inputenc}                       % if you are not using xelatex ou lualatex, replace by the encoding you are using
%\usepackage{CJKutf8}                              % if you need to use CJK to typeset your resume in Chinese, Japanese or Korean

% adjust the page margins
\usepackage[scale=0.75]{geometry}
\usepackage[latin1]{inputenc}
\usepackage[frenchb]{babel}
%\setlength{\hintscolumnwidth}{3cm}                % if you want to change the width of the column with the dates
%\setlength{\makecvtitlenamewidth}{10cm}           % for the 'classic' style, if you want to force the width allocated to your name and avoid line breaks. be careful though, the length is normally calculated to avoid any overlap with your personal info; use this at your own typographical risks...

% personal data
\name{Alexandre}{Brochu}
\address{943 Cheval-Blanc avenue}{J8R 1A2 Gatineau}{QC}
% optional, remove / comment the line if not wanted; the optional "type" of the phone can be "mobile" (default), "fixed" or "fax"
\phone[mobile]{(819) 209-9772}
\phone[fixed]{(819) 669-4016}
\email{alexbrochu1@gmail.com}
\homepage{brochu.github.io}
\social[linkedin]{Alexandre Brochu}
\social[github]{Brochu}                             
%\quote{Objectif: Obtenir un emploi dans le domaine de la programmation informatique plus sp�cifiquement en programmation de jeux vid�o pour mettre en pratique mes �tudes en informatique.}

% to show numerical labels in the bibliography (default is to show no labels); only useful if you make citations in your resume
%\makeatletter
%\renewcommand*{\bibliographyitemlabel}{\@biblabel{\arabic{enumiv}}}
%\makeatother
%\renewcommand*{\bibliographyitemlabel}{[\arabic{enumiv}]}% CONSIDER REPLACING THE ABOVE BY THIS

% bibliography with mutiple entries
%\usepackage{multibib}
%\newcites{book,misc}{{Books},{Others}}
%----------------------------------------------------------------------------------
%            content
%----------------------------------------------------------------------------------
\begin{document}
\makecvtitle
\pagestyle{empty}
\thispagestyle{plain}

\section{Education}
\cventry{2012-2015}{Bachelor degree in computer science}{Sherbrooke University}{Sherbrooke (Quebec)}{}
    {
            \begin{itemize}
                \item Final GPA of 4.06
                \item Rendering programming
                \item Artificial Intelligence
                \item Object oriented programming
            \end{itemize}
    }  % arguments 3 to 6 can be left empty
\cventry{2009-2012}{DEC in information technology}{Cegep de l'Outaouais}{Gatineau (Quebec)}{}{}
% NOT SURE ABOUT THIS, MAYBE KEEP THIS??
% \cventry{2004-2009}{Dipl�me d'�tudes secondaire}{Polyvalente de l'�rabli�re}{Gatineau (Qu�bec)}{}{ Programme micro informatique (PMI)}

\section{Work experience}
\cventry{May~2014\\Aug.~2014}{General programmer}{Square Enix Montreal}{Montreal}{QC}{
\begin{itemize}
	\item Working with Unity3D game engine
        \item Modifying the game code directly (artificial intelligence, animation, user interface)
	\item Adding functionalities to the game based on specifications
	\item Fixing problems identified by a team of testers
        \item Developing a game on a mobile platform
\end{itemize}}
\cventry{Jan.~2013\\Apr.~2013 \\ Sept.~2013\\Dec.~2013}{Web Developer}{Environment Canada}{Gatineau}{QC}{
\begin{itemize}
	\item Working in an autonomous fashion
	\item Developing a new version of a tool used to help productivity of other employees
	\item Planning the implementation of a database
	\item Writing code of a web publication tool (asp.net, MVC)
	\item Helping with web server management
\end{itemize}}
\cventry{Jan.~2012\\Aug.~2012}{Software developer}{STaCS DNA}{Ottawa}{ON}{
\begin{itemize}
	\item Working with a testing team
	\item Adding functionalities in a DNA database software
	\item Modifying software based of specifications
	\item Fixing problems identified by clients using the software
	\item Maintaining a large SQL database
	\item Internship completely done in English
\end{itemize}}

\section{Technical skills}
\cvitem{Operating systems}{Microsoft, Linux, Android}
\cvitem{Languages}{C++, C\#, Javascript, Ruby, Python, Racket, Java}
\cvitem{Development tools}{Visual Studio, Eclipse, Netbeans, Vim, CMake, GDB}
\cvitem{Version systems}{Git (Github, Bitbucket), Perforce, SVN, TeamFoundation}
\cvitem{Game engines}{Unity 3D, Unreal Engine, DromEd}
% � voir si j'ai besoin de plus
% \cvitem{category 5}{XXX, YYY, ZZZ}
% \cvitem{category 6}{XXX, YYY, ZZZ}

\section{Awards and recognitions}
\cvlistitem{Two different awards of excellence for the summer 2013 and winter 2015 semesters at the Sherbrooke University for a GPA of 4.3 (A+) at the end of each semesters.}
\cvlistitem{Best grade physics 536}
\cvlistitem{Best grade chemistry 584}
% \cvlistitem{Item 3. This item is particularly long and therefore normally spans over several lines. Did you notice the indentation when the line wraps?}

\section{Personal and university projects}
\cvlistdoubleitem{Ray tracing engine in C++}{Implementation of a reward based learning algorithm}
\cvlistdoubleitem{Solar system visualiser with OpenGL and C++}{Parallel fractal generator}
\cvlistdoubleitem{Creation of first person shooter controls with the Unity3D engine}{Creation of a game based on game mechanics observed in an existing game}

\section{Interests}
\cvitem{Video games}{I like playing video games for the experience in my free time but also to study what makes a game interesting}
\cvitem{Computer}{While studying, I built multiple computers for my close friends and myself.}
\cvitem{Music}{I have been playing electric guitar for 7 years now. I learnt all by myself and I practiced my self-education a lot while learning.}

% Publications from a BibTeX file without multibib
%  for numerical labels: \renewcommand{\bibliographyitemlabel}{\@biblabel{\arabic{enumiv}}}% CONSIDER MERGING WITH PREAMBLE PART
%  to redefine the heading string ("Publications"): \renewcommand{\refname}{Articles}
% \nocite{*}
% \bibliographystyle{plain}
% \bibliography{publications}                        % 'publications' is the name of a BibTeX file

% Publications from a BibTeX file using the multibib package
%\section{Publications}
%\nocitebook{book1,book2}
%\bibliographystylebook{plain}
%\bibliographybook{publications}                   % 'publications' is the name of a BibTeX file
%\nocitemisc{misc1,misc2,misc3}
%\bibliographystylemisc{plain}
%\bibliographymisc{publications}                   % 'publications' is the name of a BibTeX file

\clearpage
%-----       letter       ---------------------------------------------------------
% recipient data
\recipient{Snowed In Studios}{Snowed In Studios\\123 TopKekStreet\\Ottawa (ON)}
\date{19 Mars 2015}
\opening{Dear Sir or Madam,}
\closing{Yours faithfully,}
\enclosure[fichier attach�]{curriculum vit\ae{}}          % use an optional argument to use a string other than "Enclosure", or redefine \enclname
\makelettertitle

Lorem ipsum dolor sit amet, consectetur adipiscing elit. Duis ullamcorper neque sit amet lectus facilisis sed luctus nisl iaculis. Vivamus at neque arcu, sed tempor quam. Curabitur pharetra tincidunt tincidunt. Morbi volutpat feugiat mauris, quis tempor neque vehicula volutpat. Duis tristique justo vel massa fermentum accumsan. Mauris ante elit, feugiat vestibulum tempor eget, eleifend ac ipsum. Donec scelerisque lobortis ipsum eu vestibulum. Pellentesque vel massa at felis accumsan rhoncus.

Suspendisse commodo, massa eu congue tincidunt, elit mauris pellentesque orci, cursus tempor odio nisl euismod augue. Aliquam adipiscing nibh ut odio sodales et pulvinar tortor laoreet. Mauris a accumsan ligula. Class aptent taciti sociosqu ad litora torquent per conubia nostra, per inceptos himenaeos. Suspendisse vulputate sem vehicula ipsum varius nec tempus dui dapibus. Phasellus et est urna, ut auctor erat. Sed tincidunt odio id odio aliquam mattis. Donec sapien nulla, feugiat eget adipiscing sit amet, lacinia ut dolor. Phasellus tincidunt, leo a fringilla consectetur, felis diam aliquam urna, vitae aliquet lectus orci nec velit. Vivamus dapibus varius blandit.

Duis sit amet magna ante, at sodales diam. Aenean consectetur porta risus et sagittis. Ut interdum, enim varius pellentesque tincidunt, magna libero sodales tortor, ut fermentum nunc metus a ante. Vivamus odio leo, tincidunt eu luctus ut, sollicitudin sit amet metus. Nunc sed orci lectus. Ut sodales magna sed velit volutpat sit amet pulvinar diam venenatis.

Albert Einstein discovered that $e=mc^2$ in 1905.
$$ e=\lim_{n \to \infty} \left(1+\frac{1}{n}\right)^n $$

\makeletterclosing
\end{document}
